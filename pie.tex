%%%%%%%%%%%%%%%%%%%%%%%%%%%%%%%%%%%%%%%%%%%%%%%
% Implements a pie chart for pgfplots         %
% graphics                                    %
%                                             %
% author:  Enys Mones                         %
% date:    2016-12-07                         %
% version: 0.1                                %
%%%%%%%%%%%%%%%%%%%%%%%%%%%%%%%%%%%%%%%%%%%%%%%

% Imports
\usepackage{tikz}
\usepackage{pgfplots}


% Core of pie chart
\tikzstyle{pie chart}=[
    slice/.style={line cap=round, line join=round, very thick,draw=white},
    slice type/.style 2 args={
        ##1/.style={fill=##2},
        values of ##1/.style={}
    }
]
\pgfdeclarelayer{background}
\pgfdeclarelayer{foreground}
\pgfsetlayers{background,main,foreground}


% Draws a pie chart
%
% arg 1: additional style for the pie chart
% arg 2: slice sizes and style
\newcommand{\pie}[2][]{
    \begin{scope}[pie chart,#1]
    \pgfmathsetmacro{\curA}{90}
    \pgfmathsetmacro{\r}{1}
    \def\c{(0,0)}
    \edef\sum{0}
    \foreach \v/\s in{#2}{
      \pgfmathparse{\sum+\v}
      \xdef\sum{\pgfmathresult}
    }
    \foreach \v/\s in{#2}{
        \pgfmathsetmacro{\deltaA}{\v/\sum*360}
        \pgfmathsetmacro{\nextA}{\curA + \deltaA}
        \pgfmathsetmacro{\midA}{(\curA+\nextA)/2}

        \path[slice,\s] \c
            -- +(\curA:\r)
            arc (\curA:\nextA:\r)
            -- cycle;
        \pgfmathsetmacro{\d}{max((\deltaA * -(.5/50) + 1) , .5)}

        %\begin{pgfonlayer}{foreground}
        %\path \c -- node[pos=\d,pie values,values of \s]{$\v\%$} +(\midA:\r);
        %\end{pgfonlayer}

        \global\let\curA\nextA
    }
    \end{scope}
}
